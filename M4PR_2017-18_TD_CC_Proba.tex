\documentclass[a4paper,11pt,reqno]{amsart}
\usepackage{M4ParcRenf}

% \solutionstrue

\begin{document}

% ==================================
\hautdepage{

\ifsolutions{Solutions de l'interrogation}\else{Interrogation}\fi\par\normalfont\normalsize
14 février 2018\\{[ durée: 1 heure ]}\par
}
% ==================================
\ifsolutions\else
% {\fontencoding{U}\fontfamily{futs}\selectfont\char 66\relax}
\tikz[baseline=(e.base)]{\NoAutoSpacing\node(e){!};\draw[red,ultra thick,line join=round,yshift=-.15ex](90:1em)--(210:1em)--(330:1em)--cycle;}
\textbf{Les documents et les calculatrices ne sont pas autorisés.}

\hfill
\fi

%-----------------------------------
\begin{exo}

  Dans un jeu de $3$ dés, il y un des dés qui est truqué, et qui ne tombe jamais sur $6$ (les cinq autres faces sont équiprobables). On jette les $3$ dés et on observe.
  \begin{enumerate}
    \item Quel est l'espace de probabilité que vous considérez? Quelle est la probabilité d'un événement élémentaire?

    \item Quelle est la probabilité d'obtenir $3$ dés identiques?

    \item Quelle est la probabilité d'obtenir exactement $2$ dés identiques?
  \end{enumerate}
\end{exo}

\begin{solution}
  \begin{enumerate}
    \item L'espace de probabilité considéré est $\Omega = \{1,\ldots,5\} \times \{1,\ldots,6\}^{2}  $ avec la probabilité uniforme. Comme le nombre d'éléments de $\Omega$ est $\#\Omega = 5 \times 6^{2}=180$, la probabilité d'obtenir une configuration particulière (un événement élémentaire) est de $\frac{1}{180}$.

    \item L'événement $3 \times 6$ est impossible, et pour tout autre $i=1,\ldots,5$, la probabilité d'obtenir $3 \times i$ est de $\frac{1}{180}$. Donc la probabilité d'obtenir $3$ dés identiques est de $5\times\frac{1}{180} = \frac{1}{36}$.

    \item Le nombre de configurations avec trois résultats différents est $5 \times 5 \times 4 = 100$. Donc la probabilité d'avoir les trois dés différents est $\frac{100}{180}=\frac{5}{9}$. Ainsi finalement la probabilité d'avoir deux dés identiques est $1-\frac{1}{36}-\frac{5}{9} = \frac{15}{36} = \frac{5}{12}$.
  \end{enumerate}
\end{solution}

%-----------------------------------
\begin{exo}

  Un enseignant pose une question à un étudiant qui doit répondre par oui ou non. On sait que l'étudiant connaît la bonne réponse avec probabilité $p$ et dans ce cas il donne la bonne réponse. Si l'étudiant ne connaît pas la réponse, il répond alors au hasard «oui» ou «non» avec probabilité \sfrac12 chacun.

  En écrivant proprement les événements en jeu, calculer les probabilités:
  \begin{enumerate}
    \item que l'étudiant donne la bonne réponse,

    \item que l'étudiant connaisse la bonne réponse sachant qu'il a répondu correctement.
  \end{enumerate}
\end{exo}

\begin{solution}

\begin{enumerate}
  \item On note $C=$«l'étudiant connaît la bonne réponse» et $R$=«l'étudiant donne la bonne réponse». D'après l'énoncé on a $P(C)=p$, $P(R|C)=1$ et $P(R|\non{C})=\frac12$. Ainsi $P(R) = P(R|C)P(C) + P(R|\non{C})P(\non{C}) = 1p+\frac{1}{2}(1-p)=\frac{1+p}{2}$.
  \item Nous avons $P(C|R) = \frac{P(R|C)P(C)}{P(R)} = \frac{2p}{1+p}$.
\end{enumerate}

\end{solution}

%-----------------------------------
\begin{exo}

   Dans un lot de 100 composants électroniques, il y a deux composants défectueux. On prélève au hasard sans remise $n$ composants dans ce lot et on note $X$ le nombre de composants défectueux parmi les $n$ prélevés.
  \begin{enumerate}
    \item On suppose que $2 \le n \le 98.$ Donner la loi de $X$.
    \item Quelle est la loi de $X$ si $n=100$ ?
    \item Je choisis un composant au hasard. Quelle est la probabilité qu'il soit défectueux ?
    \item En déduire la loi de $X$ si $n=1$.
    \item En déduire aussi la loi de $X$ si $n=99$.
  \end{enumerate}
\end{exo}

\begin{solution}
  \begin{enumerate}
    \item Comme il s'agit d'un tirage sans remise de $n$ éléments parmi $100$ et $X$ compte l'apparition de $2$ de ces éléments, on a $X \sim \mathcal{HG}(100,n,2)$ avec $X \in \{0,1,2\}$ et $P(X=k) = \frac{C_{2}^{k}C_{98}^{n-k}}{C_{100}^{n}}$ pour $k=0,1,2$. Cette formule marche pour tout $n \in \ldbrack 0,100 \rdbrack$, non seulement pour $2 \leq n \leq 98$.
    \item Si on prend les $100$ composants électroniques on sait que $X=2$, donc $X$ est une constante et $P(X=2)=1$. On retrouve ce résultat avec la formule générale car $P(X=k) = \frac{C_{2}^{k}C_{98}^{100-k}}{C_{100}^{100}}$ et $C_{98}^{100}=C_{98}^{99}=0$, donc $P(X=0)=P(X=1)=0$ et $P(X=2) = \frac{C_{2}^{2}C_{98}^{98}}{C_{100}^{100}}=1$.
    \item $P(\text{«un composant défectueux»}) = \frac{2}{100}$.
    \item Pour $n=1$ nous avons $X \in \{0,1\}$ et d'après la question précédente on a $P(X=1)=\frac{2}{100}$ et $P(X=0)=1-P(X=1)=\frac{98}{100}$. On peut retrouver ce résultat par la formule générale : $P(X=0) = \frac{C_{2}^{0}C_{98}^{1}}{C_{100}^{1}} = \frac{98}{100}$, $P(X=1) = \frac{C_{2}^{1}C_{98}^{0}}{C_{100}^{1}} = \frac{2}{100}$ et $P(X=2) = \frac{C_{2}^{2}C_{98}^{-1}}{C_{100}^{1}} = 0$.
    \item Si on pose $Y=$«le nombre de composants défectueux parmi les $100-n$ non prélevés», nous avons $X+Y=2$ et $Y \sim \mathcal{HG}(100,100-n,2)$. Ainsi pour $n=99$ on a $Y \sim \mathcal{HG}(100,1,2)$ et donc d'après la question précédente $P(X=2)=P(Y=0)=\frac{98}{100}$, $P(X=1)=P(Y=1)=\frac{2}{100}$ et $P(X=0)=P(Y=2)=0$. On peut, comme dans les questions précédentes, retrouver ce résultat par la formule générale.
  \end{enumerate}
\end{solution}

\end{document}


